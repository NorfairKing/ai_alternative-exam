\documentclass[alternative-exam.tex]{subfiles}
\begin{document}

\chapter{Kater}
\section{Vraag}
Op een ochtend word je met barstende hoofdpijn wakker in je bed op kot. Je herinnert je dat je op een kroegentocht bent geweest op de grote markt. Je zou de grote markt in wijzerzin afgaan beginnende bij caf\'e Manger \footnote{Ter herinnering: De caf\'e's op de oude markt in volgorde zijn "Manger, Alegria, Rock Caf\'e, Cafe 33}. Je herinnert je nog dat je in caf\'e Manger en in de Alegria bent geweest, over de gebeurtenissen daarna weet je niets meer. Je belt naar je beste vriend en die zegt dat hij met jou in het Rock Caf\'e aan de toog heeft gezeten. Na het telefoontje besef je dat als je je beste vriend hebt gezien, hij je dan zeker nog naar \'e\'en volgend caf\'e meegenomen. Je vraagt je af of je in Caf\'e 33 bent geraakt.

Als goede informaticus merk je meteen op dat dit probleem op te lossen valt met een automatisch redeneersysteem. Probeer te bewijzen dat je in De Kroeg bent geraakt. Als hulpmiddel krijg je het alfabet. Let goed op dat je ook de impliciete logische beweringen formuleert.

Als alfabet gebruik je voor deze opgave de predicaten in figuur \ref{alfabet}. Bovendien korten we enkele elementen af zoals beschreven in figuur \ref{cafes}.
\begin{figure}[H]
\centering
\begin{figure}[H]
\centering
\begin{tabular}{c | l | l}
Predicaat & Betekenis & Afkorting\\
\hline
$Cafe(c)$ & $c$ is een Caf\'e. & $C(c)$\\
$Persoon(p)$ & $p$ is een Persoon. & $P(p)$\\
$Geweest(c)$ & Je ben in caf\'e $c$ geraakt.) & $Gw(c)$\\
$Gezien(p,c)$ & Je bent door persoon $p$ in caf\'e $c$ geweest & $G(p,c)$\\
$LinksVan(c_1,c_2)$ & Caf\'e $c_1$ staat links van Caf\'e $c_2$. & $L(c_1,c_2)$\\
\end{tabular}
\caption{Alfabet}
\label{alfabet}
\end{figure}
\begin{tabular}{c | c}
Element &  Afkorting\\
\hline
Caf\'e Manger & CM\\
Alegria & AG\\
Rock Caf\'e & RC\\
Caf\'e 33 & C3\\
je beste vriend & BV\\
\end{tabular}
\caption{Elementen}
\label{cafes}
\end{figure}


\section{Modeloplossing}
\subsection{Vertaling}
We vertalen nu alle mogelijke gegevens uit de opgave. De eerste vier beweringen zijn impliciet (en evident), maar moeten zeker worden geformuleerd.
\begin{itemize}
\item De opgesomde caf\'e's zijn caf\'e's en je beste vriend is een persoon.
\[ C(CM)\wedge C(AG)\wedge C(RC)\wedge C(C3)\wedge C(DK) \]
\[ P(BV)\]
\item De caf\'es staan in volgorde.
\[
L(CM,AG) \wedge L(AG,RC) \wedge L(RC,C3)
\]
\item Als iemand je in een caf\'e heeft gezien, ben je daar geweest.
\[
(\forall a)(C(a) \Rightarrow ((\exists p) (P(p) \wedge G(p,a)) \Rightarrow Gw(a)))
\]
\item Je bent in wijzerzin 'te werk' gegaan.
\[
(\forall b)(( C(b) \wedge Gw(b)) \Rightarrow ((\forall c)(( C(c) \wedge L(c,b)))\Rightarrow Gw(c)))
\]
\item Je bent in Caf\'e Manger en de Alegria geweest.
\[
Gw(CM) \wedge Gw(AG)
\]
\item Je beste vriend zegt dat hij met jou in het Rock Caf\'e aan de toog heeft gezeten.
\[
G(BV,RC)
\]
\item Als je je beste vriend hebt gezien, heeft hij je zeker dan nog naar \'e\'en volgend caf\'e meegenomen.
\[
(\forall d) ((C(d) \wedge G(BV,d)) \Rightarrow ((\exists e) (C(e) \wedge Gw(e) \wedge L(d,e))))
\]
\end{itemize}
Nu proberen we het volgende te bewijzen.
\[
Gw(C3)
\]

\subsection{Implicatieve normaalvorm}
We zetten nu alle vertaalde formules om naar implicatieve normaalvorm.
Wanneer kwantoren weggelaten kunnen worden, worden ze eerst in het rood gezet. De bekomen logische zinnen (dit kunnen er meerdere zijn) worden achteraf nog samengevat. Tenslotte voegen we ook negatie van de hypothese toe en zetten deze ook om naar implicatieve normaalvorm.
\begin{itemize}
\item $C(CM)\wedge C(AG)\wedge C(RC)\wedge C(C3)$
\[
\left\{
\begin{array}{c c c}
C(CM) &\leftarrow& True\\
C(AG) &\leftarrow& True\\
C(RC) &\leftarrow& True\\
C(C3) &\leftarrow& True\\
\end{array}
\right.
\]

\item $P(BV)$
\[
P(BV) \leftarrow True\\
\]

\item $L(CM,AG) \wedge L(AG,RC) \wedge L(RC,C3)$
\[
\left\{
\begin{array}{c c c}
L(CM,AG) &\leftarrow& True\\
L(AG,RC) &\leftarrow& True\\
L(RC,C3) &\leftarrow& True\\
\end{array}
\right.
\]

\item $(\forall a)(C(a) \Rightarrow ((\exists p) (P(p) \wedge G(p,a)) \Rightarrow Gw(a)))$
\begin{itemize}
\item $ {\color{red}(\forall a)} (\neg C(a) \vee (\neg ((\exists p)(P(p)\wedge G(p,a))) \vee Gw(a)))$

\item $\neg C(a) \vee ((\forall p)\neg(P(p)\wedge G(p,a))) \vee Gw(a)$

\item $\neg C(a) \vee (\forall p)(\neg P(p)\vee \neg G(p,a)) \vee Gw(a)$

\item $(\forall p) \neg C(a) \vee \neg P(p) \vee \neg G(p,a) \vee Gw(a)$

\item ${\color{red}(\forall p)} \neg C(a) \vee \neg P(p) \vee \neg G(p,a) \vee Gw(a)$

\[
Gw(a)\leftarrow C(a) \wedge P(p) \wedge G(p,a)
\]
\end{itemize}
\item $(\forall b)(( C(b) \wedge Gw(b)) \Rightarrow ((\forall c)(( C(c) \wedge L(c,b)))\Rightarrow Gw(c)))$
\begin{itemize}
\item ${\color{red}(\forall b)}( C(b) \wedge Gw(b)) \Rightarrow ((\forall c)(( C(c) \wedge L(c,b)))\Rightarrow Gw(c))$

\item $\neg( C(b) \wedge Gw(b)) \vee ((\forall c)\neg(( C(c) \wedge L(c,b)))\vee Gw(c))$

\item $\neg C(b) \vee \neg Gw(b) \vee (\forall c)( \neg C(c) \vee \neg L(c,b))\vee Gw(c)$

\item ${\color{red}(\forall c)} \neg C(b) \vee \neg Gw(b) \vee \neg C(c) \vee \neg L(c,c)\vee Gw(c)$
\[
 Gw(c)\leftarrow C(b) \wedge Gw(b) \wedge C(c) \wedge L(c,b)
\]
\end{itemize}

\item $Gw(CM) \wedge Gw(AG)$
\[
\left\{
\begin{array}{c c c}
Gw(CM) &\leftarrow& True\\
Gw(AG) &\leftarrow& True\\
\end{array}
\right.
\]

\item $G(BV,RC)$
\[
G(BV,RC) \leftarrow True\\
\]

\item $(\forall d) ((C(d) \wedge G(BV,d)) \Rightarrow ((\exists e) (C(e) \wedge Gw(e) \wedge L(d,e)))$
\begin{itemize}
\item ${\color{red}(\forall d)}((C(d) \wedge G(BV,d)) \Rightarrow ((\exists e) (C(e) \wedge Gw(e) \wedge L(d,e))))$

\item $\neg (C(d) \wedge G(BV,d)) \vee (\exists e) (C(e) \wedge Gw(e) \wedge L(d,e))$

\item $\neg C(d) \vee \neg G(BV,d) \vee {\color{red} (\exists e)} (C(e) \wedge Gw(e) \wedge L(d,e))$

\item $\neg C(d) \vee \neg G(BV,d) \vee (C(A) \wedge Gw(A) \wedge  L(d,A))$

\item 
$(C(A)\vee \neg C(d) \vee \neg G(BV,d)) \wedge (Gw(A)\vee  \neg C(d) \vee \neg G(BV,d)) \wedge (L(d,A) \vee \neg C(d) \vee \neg G(BV,d))$

\[
\left\{
\begin{array}{c c c}
C(A)\leftarrow  C(d) \wedge  G(BV,d)\\
Gw(A) \leftarrow  C(d) \wedge  G(BV,d)\\
L(d,A) \leftarrow C(d) \wedge G(BV,d)\\
\end{array}
\right.
\]
\end{itemize}

\item $\neg Gw(C3)$
\[
False \leftarrow Gw(C3)
\]
\end{itemize}

\subsection{Resolutie}
De resolutie is eigenlijk helemaal niet zo moeilijk, maar je kan heel lang bezig zijn als je het te ver gaat zoeken.

\begin{enumerate}

\item $Gw(A) \leftarrow C(d) \wedge G(BV,d)$
\begin{itemize}
\item $False \leftarrow Gw(C3)$
\item Resolutie: $\{ A/C3\}$
\end{itemize}

\item $False \leftarrow C(d) \wedge G(BV,d)$
\begin{itemize}
\item $G(BV,RC) \leftarrow True$
\item Resolutie: $\{ d/RC\}$
\end{itemize}

\item $False \leftarrow C(RC)$
\begin{itemize}
\item $C(RC) \leftarrow True$
\end{itemize}

\item $False \leftarrow True$
\end{enumerate}
We vinden een contradictie in stap $4$. Omdat we tot een contradictie komen door de negatie van de hypothese toe te voegen aan de gegevens kunnen we besluiten dat de hypothese geldt in de gegeven structuur.



\end{document}